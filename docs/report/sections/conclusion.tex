\section{Conclusion}
In this project, two related but differentiated task were carried out and described, the creation of a knowledge graph from a relational database and the linking and completion of a knowledge graph with the use of other external graphs. 

Regarding the first task, it can be concluded that the difficulty of generating a graph from a relational database will depend greatly on the scheme complexity. 
As the mapping tool requires a tabular input, such a table or the selected attributes of a query, obtaining the information necessary to generate a relation between two different type of entities could implicate the creation of complex queries.  

Nonetheless, other aspect to remark is the superiority of the knowledge graph to represent some relations between entities with respect to the relational databases.
A clear example could be the ``member’’ relation that exist between music groups and musicians, that is direct in the knowledge graph generated whereas to extract this information in the database the join of three different tables is required \footnote{\href{https://github.com/adrigrillo/music_kg/blob/master/queries/sql/query-artist.sql}{SQL queries} used to generate the mapping of the artists}.  

With respect the interlinking and, specially, the completion of the graph with external sources, this project highlights the difficulty of finding a generalized way to relate entities of different datasets. 
This complexity not only comes from the lack of common vocabularies and structures among knowledge graphs but also from the inherent nature of the things, with a clear example in the existence of two different kind of entities with the same name. 