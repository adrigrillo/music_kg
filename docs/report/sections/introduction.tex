\section{Introduction}
With regards to the distribution of the music, internet has become the platform that provides the greatest revenues, being 2017 the first year that it accounted over the half of the total revenues generated \citep{music_report_2018}.
This platform presents a great opportunity to all kind of artist but, at the contrary of what happened in the past, the entry barrier is much lower.
Nowadays, amateur and alternative artists can upload their music to the cloud to make themselves known without having to depend from a record and start their musical career. 

However, in order to be discovered by a broader audience, the information about these artists has to be complete and correct.
Only in that way, a better position in the search engines and in the recommendations will be achieved.
Although some of this data depends of closed services where the music can be uploaded, there also exist public domain knowledge platforms like Wikipedia \citep{wikipedia}, Wikidata \citep{wikidata}, DBpedia \citep{dbpedia} or, more specialized ones, like MusicBrainz \citep{musicbrainz} that facilitates the publication of this information and which are used to generate results for the users. 

In any case, including a complete and consistent information across the different platforms requires a dedication that most of this kind of artists cannot afford, leading to situations where the data differs between websites or, directly, is missing. 
In this project, the consistency of the information among different public domain platforms and the completion between each other will be assessed through the use of a knowledge graphs. 