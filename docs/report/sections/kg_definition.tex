\section{Definition of Knowledge Graph}
When it comes to define what is a Knowledge Graph (KG), both in the academic and business world, there is not a consensus about the characteristics required to fulfil the definition.
As it is explained in \citet{Ehrlinger2016} a great variety of definitions have arisen in the last years with the developments made in this field.
In this project, a knowledge graph will be created from a relational database and will be compared and completed with the information contained in other known open knowledge graphs. 

We are going to define a knowledge graph as an information storage system where the content is described and interrelated forming a network that can be seen as a graph.
Following the mathematical definition of graph, in a knowledge graph there will be vertices, which in a KG will represent the entities, and edges, which will represent the relation between entities.  

An entity is the representation of an independent thing, like a person, a place or an object, that will belong to a type defined by an ontology and will have some properties that will describe it.
An example could be the entity ``Dave Grohl'', whose type will be ``musician'' that will be defined by an ontology and will have some properties like the name and surname that will describe the entity. 

On the other hand, an edge will represent the relationships between different entities contained in the knowledge graph.
These relationships will be also defined by an ontology and are the only way of directly relate two independent entities.
Continuing with the previous example, ``Dave Grohl'' will be associated with the entity ``Foo Fighters'' with the relation ``is member of'', that in this case, also generates the inverse relation ``Foo Fighters'' has ``Dave Grohl'' as a member of the group. 

The use of the relationships will also allow the connection of the information contained within the knowledge graph with other external sources, in a process known as interlinking.
Moreover, apart from representing information, a KG will allow the extraction of new knowledge through the exploitation of the relationship between entities and external sources. 

In this project, the Resource Description Framework (RDF) will be used to generate the graph.
The main difference of this kind of graph regarding property graphs is that the relationships between entities do not contain properties.  